\section{Autres}
\subsection{Commandes}
\paragraph{netcat} (\verb!nc!) : Couteau suisse du TCP/IP. Permet de scanner des ports
\paragraph{nmap} : Analyse des ports ouverts
\paragraph{ssh} : Connexion à un système par interpréteur de commande
\paragraph{dd} : copie byte à byte entre des streams. (\verb!sudo dd if=/dev/zero/!\\\verb!of=/dev/null bs=512 count=100 seek=16!)
\paragraph{parted} : création / modification de partiations (\verb!sudo parted /dev/sdb mklabel msdos!
\paragraph{mkfs.ext4} : commandes ext4 pour créer / modifier une partition

\subsection{Définitions}
\paragraph{Honeypot} : "Pot de miel" ou leurre pour faire croire qu'un système non-sécurisé est présent (à tord)
\paragraph{Toolchain} : Codes sources et outils nécessaires pour générer une image éxécutable (sur un système embarqué)
\paragraph{Kernel} : Coeur Linux (avec le format u-boot)
\paragraph{Rootfs} : Root Filesystem (avec tous les dossiers et outils utilisés par Linux)
\paragraph{Usrfs} : User Filesystem (applications spécifiques à l'utilisation du système embarqué)
\paragraph{Buildroot} : Ensemble de makefiles et patchs qui simplifient et automatisent la création d'un Linux pour système embarqué
\paragraph{uClibc} : Librairie c de base similaire à glibc mais plus compacte (pour systèmes MMU-less)
\paragraph{Busybox} : Binaire unique qui contient toutes les commandes de base (ls, cat, mv)