% Firewall (DONE)
\section{Firewall \texttt{iptables}}
\verb+iptables -t table -COMMAND chain ... -j TARGET+\\
Le kernel doit être configuré pour activer netfilter.

Un hook est une étape lors du passage d'une trame dans le stack de protocoles. Le framework netfilter sera appelé à chaque hook (combinaison chain-table).
% ----------
\subsection{Chains}
Quand un paquet arrive il traverse différentes chains contenues dans différentes tables:
\begin{itemize}
    \item INPUT: le paquet est pour l'hôte local
    \item OUTPUT: le paquet est émis par l'hôte local
    \item FORWARD: le paquet arrive sur une interface (eth0) et est transmis à une autre (eth1).
    \item PREROUTING: pour modifier paquets dès réception. (NAT)
    \item POSTROUTING: pour modifier paquets juste avant émission. (NAT)
\end{itemize}
% ----------
\subsubsection{Tables}
\begin{itemize}
    \item filter: principalement utilisée (default).
    \item mangle: alternation-modification particulière sur des paquets.
    \item nat: consultée lorsqu'un paquet créé une nouvelle connexion (Network Address Translation).
\end{itemize}
% ----------
\subsection{Features}
\begin{enumerate}
    \item \textbf{Stateless packet} firewall (table filter et ACCEPT, DROP, REJECT). Permet de protéger au niveau réseau (bloquage d'une ip, d'un port, etc.). $\rightarrow$ paquets analysés de manière individuelle.
    \item \textbf{Stateful} firewall. Permet de protéger au niveau du paquet en fonction du contexte (précédents paquets). Il est possible d'accepter des paquets venant de l'extérieur seulement s'ils sont des réponses à des requêtes venant de l'intérieur.
    \begin{itemize}
        \item Utilisation de connection tables pour traiter les différentes parties des protocoles.
        \item NEW : Nouveau paquet qui n'est pas lié à une connexion active
        \item ESTABLISHED : Une connexion passe de NEW à ESTABLISHED losrque la connexion est validée par la direction opposée
        \item RELATED : Paquets qui ne font pas partie d'une connexion existante mais qui sont liés à une autre. (Par exemple réponses ICMP pour une communication FTP).
    \end{itemize}
    \item Translation d'adresses / ports (NAT)
    \item API pour autres applications
\end{enumerate}
% ----------
\subsection{NFQUEUE}
NFQUEUE permet de transmettre un paquet au userspace (hors du noyau).
% ----------
\subsection{knockd}
Réside dans le user space. Configuration dynamique du firewall Netfilter avec des commandes \verb!iptables!. Par exemple ouvertures de port spécifiques lorsque des séquences de \textit{Port Knocking} (PK) sont reconnues.
% ----------
\subsection{fwknop (FireWall KNock OPerator)}
Identique que \verb+knockd+ mais il utilise le contenu des paquets TCP/UDP (frame SPA - Single Packet Authorization). Si un SPA est valide, le pare-feu est ouvert pendant un certain temps (par exemple 30s).
